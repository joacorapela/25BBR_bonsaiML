
\documentclass[11pt]{article}
\usepackage[margin=2cm]{geometry}
%% Language and font encodings
\usepackage[english]{babel}
\usepackage[utf8x]{inputenc}
\usepackage[T1]{fontenc}
\usepackage[scaled=0.92]{helvet} \renewcommand{\familydefault}{\sfdefault}
\usepackage{graphicx}
\usepackage{hyperref}
\usepackage[numbers]{natbib}
\usepackage{bibunits}
\usepackage{longtable}
\usepackage{titlesec}
\hypersetup{
    colorlinks = true,
    linkcolor = {green},
}
\usepackage{verbatim}
\usepackage{lineno}
\usepackage{array}
\usepackage[most]{tcolorbox}

\parindent 0pt
\parskip 1ex

\newenvironment{instruction}{%
    \begin{tcolorbox}[breakable,colback=red!5,colframe=red,title=Instruction]%
	}{%
    	\end{tcolorbox}%
	}

% begin add subsubsubsection
\titleclass{\subsubsubsection}{straight}[\subsection]

\newcounter{subsubsubsection}[subsubsection]
\renewcommand\thesubsubsubsection{\thesubsubsection.\arabic{subsubsubsection}}
\renewcommand\theparagraph{\thesubsubsubsection.\arabic{paragraph}} % optional; useful if paragraphs are to be numbered

\titleformat{\subsubsubsection}
  {\normalfont\normalsize\bfseries}{\thesubsubsubsection}{1em}{}
\titlespacing*{\subsubsubsection}
{0pt}{3.25ex plus 1ex minus .2ex}{1.5ex plus .2ex}

\makeatletter
\renewcommand\paragraph{\@startsection{paragraph}{5}{\z@}%
  {3.25ex \@plus1ex \@minus.2ex}%
  {-1em}%
  {\normalfont\normalsize\bfseries}}
\renewcommand\subparagraph{\@startsection{subparagraph}{6}{\parindent}%
  {3.25ex \@plus1ex \@minus .2ex}%
  {-1em}%
  {\normalfont\normalsize\bfseries}}
\def\toclevel@subsubsubsection{4}
\def\toclevel@paragraph{5}
%\def\toclevel@paragraph{6}
\def\toclevel@subparagraph{6}
\def\l@subsubsubsection{\@dottedtocline{4}{7em}{4em}}
\def\l@paragraph{\@dottedtocline{5}{10em}{5em}}
\def\l@subparagraph{\@dottedtocline{6}{14em}{6em}}
\makeatother

\setcounter{secnumdepth}{4}
\setcounter{tocdepth}{4}
% end add subsubsubsection

% \linenumbers

\title{Enabling Naturalistic, Long-Duration and Continual Experimentation with
Advanced Machine Learning Methods}

\begin{document}
\defaultbibliography{longDurationExperimentation,neuroEthology,machineLearning,signalProcessing,bonsai,spikeSorting}

\tableofcontents

\pagebreak

\section{Summary}

\begin{instruction}
Word limit: 550

In plain English, provide a summary we can use to identify the most suitable
experts to assess your application.

We usually make this summary publicly available on external-facing websites,
therefore do not include any confidential or sensitive information. Make it
suitable for a variety of readers, for example:

\begin{itemize}
	\item opinion-formers
	\item policymakers
	\item the public
	\item the wider research community
\end{itemize}

Guidance for writing a summary

Clearly describe your proposed work in terms of:

\begin{itemize}
	\item context
	\item the challenge the project addresses
	\item aims and objectives
	\item potential applications and benefits
\end{itemize}

\end{instruction}

% \input{summary}

\pagebreak
\section{Core team}

\begin{instruction}
List the key members of your team and assign them roles from the following:

\begin{itemize}
	\item project lead (PL)
	\item project co-lead (UK) (PcL)
	\item specialist
	\item grant manager professional enabling staff
	\item research and innovation associate
	\item technician
	\item researcher co-lead (RcL)
\end{itemize}

Only list one individual as project lead.

UKRI has introduced a new addition to the ‘Specialist’ role type. Public
contributors such as people with lived experience can now be added to an
application.

Find out more about UKRI’s core team roles in funding applications.

\end{instruction}

% \input{coreTeam}

\pagebreak
\section{Application questions}

\subsection{Vision}

\begin{instruction}

Word limit: 1,700

What are you hoping to achieve with your proposed resource?

What the assessors are looking for in your response

Explain how your proposed resource:

\begin{enumerate}

	\item is of excellent quality and importance within or beyond the field(s)
or area(s), and has the potential to advance current understanding, or generate
new knowledge, thinking or discovery within or beyond the field or area

	\item is timely given current trends, context, and needs

	\item impacts world-leading research, society, the economy, or the
environment

\end{enumerate}

Include the following in your statement:

\begin{enumerate}

	\item the uniqueness and expected added value of the proposed resource to
the UK bioscience research community and infrastructure landscape

	\item how the resource relates to past and current resources in the subject
area in both the UK and abroad

	\item full details of the resource and an overview of the associated
objectives.  Details on how these objectives are delivered should be included
in the ‘Approach’ section

	\item a description of the types of research that will be enabled by the
resource

	\item consideration of the potential impact on the scientific community and
other possibly dependent resources if the resource did not exist

\end{enumerate}

In your vision, you should also clearly identify which of the following categories
your proposed resource falls under, and expand on the relevant points raised
below:

\begin{enumerate}

	\item establishment of a new and innovative resource that will be
beneficial to a broader BBSRC user base. Explain why a new resource is needed
and what unique and important features it will offer

	\item maturation and subsequent maintenance of a project-based resource
into a community-based one. Briefly explain the background to the resource,
current usage, proposed changes and the benefits this will lead to for the
research community

	\item further development or essential maintenance of an existing community
resource, with well-established access mechanisms. Explain current usage and
how this project will increase its relevance, quality and utility. For example,
by enabling the resource to support FAIR (findable, accessible, interoperable,
reusable) principles, or by enabling new uses, such as metadata enrichment for
machine learning and AI approaches

	\item association, or integration, of distinct resources. Explain current
usage and how the proposed plans will create an upgraded resource with a
greater value than the sum of the parts

\end{enumerate}

References may be included within this section.
You may demonstrate elements of your responses in visual form if relevant.
Further details are provided in the Funding Service.
\end{instruction}

% \begin{bibunit}[unsrtveryabbrv]
% % Bonsai

\href{https://bonsai-rx.org/}{Bonsai} is a free and open-source visual reactive
programming language widely used for experimental control in neuroscience.
Designed for performance, flexibility, and ease of use, Bonsai enables
scientists with little or no programming background to build high-performance
data acquisition and control systems. With more than 7,000 downloads annually,
nearly 100 citations per year of the core paper, and over 1,000 new users in
2024 alone, Bonsai has become the most widely adopted software platform in
systems neuroscience. Its success demonstrates the transformative role that
sustainable, community-driven research software can play in accelerating
discovery.

% Bonsai.ML
% BBSRC grant

A central priority of UKRI is the application of
\href{https://www.ukri.org/what-we-do/browse-our-areas-of-investment-and-support/artificial-intelligence-in-bioscience/}{artificial
intelligence in bioscience}. In 2022 we recognised that integrating machine
learning (ML) into Bonsai could be transformative for experimental
neuroscience. With BBSRC support
(\href{https://gow.bbsrc.ukri.org/grants/AwardDetails.aspx?FundingReference=BB\%2FW019132\%2F1}{BB/W019132/1}),
we developed Bonsai.ML, which extends Bonsai with state-of-the-art ML methods.
These include
\href{https://bonsai-rx.org/machinelearning/examples/examples/LinearDynamicalSystems/README.html}{Linear
Dynamical Systems},
\href{https://bonsai-rx.org/machinelearning/examples/examples/HiddenMarkovModels/README.html}{Hidden
Markov Models},
\href{https://bonsai-rx.org/machinelearning/examples/examples/Torch/NeuralNetsTrainedOnline/README.html}{Deep
Neural Networks}, and a
\href{https://bonsai-rx.org/machinelearning/examples/examples/PointProcessDecoder/DecodePositionFromHippocampusSortedUnits/README.html}{Point-Process
Decoder}. Embedding these models directly in Bonsai's reactive programming
environment enables adaptive, data-driven experimental designs that were
previously out of reach for many laboratories.

% Challenges

However, adding even the best ML methods to Bonsai.ML is not enough to guarantee
their uptake, since most Bonsai users have little or no training in ML, and
cannot immediately appreciate how these methods can aid their experimentation.
%
% This challenge is not unique to Bonsai.ML, but is common to all software
% seeking to provide ML functionality to non-ML-specialists.
%
% If successful, our proposed solution will serve as a role model to all these
% software.

% Solutions

We will accelerate the uptake and impact of Bonsai.ML by working hand-in-hand
with experimental neuroscientists to address challenging neuroscience problems
with Bonsai.ML, and create detailed use cases demonstrating transformative
applications of Bonsai.ML to neuroscience experimental control.
%
% By reading or watching use cases of applications of Bonsai.ML methods in
% intelligent experimental control, experimental neuroscientists could realise
% the utility of these methods for their own experiments and adopt Bonsai.ML.

These collaboration will mostly use machine learning methods already integrated
into Bonsai.ML, but some of them will require new methods that we will add to
Bonsai.ML and expand its functionality.

\subsubsection{Collaborations}

Collaborations with experimental groups are essential for the creation of
machine learning methods targeted to biological applications. These
collaborations benefit both the experimental and the methods development
partners.
%
The former partner benefits from the machine learning expertise to interrogate its experimental data.
%
The latter partner benefits from state-of-the-art experimental
data, and from biological expertise on the significance and potential of applications of
the machine learning method in biological problems.
%
These collaborations are essential to maximise the impact of
\href{https://www.ukri.org/what-we-do/browse-our-areas-of-investment-and-support/artificial-intelligence-in-bioscience/}{artificial
intelligence in bioscience}.

During the creation of Bonsai.ML, we attempted a few collaborations with
experimental neuroscience groups as, from our experience disseminating Bonsai,
we learned that collaborations are essential to maximize uptake of software for
neuroscientists.
%
In some cases we obtained promising initial results, but achieving the full
goals of the collaboration required further developments, that we did not have
time to pursue.
%
In other collaborations, initial results were not satisfactory, but we could
not dedicate further time to understand the problems that yielded poor
performance, since we were focused on integrating machine learning
functionality into Bonsai.ML.

Now that Bonsai.ML has been created and that it contains core real-time machine
learning methods, we will focus on developing collaborations with experimental
neuroscience groups, use methods already integrated into Bonsai.ML, or develop
new ones, to jointly tackle their intelligent experimental control problems,
and use the results of these collaborations to create a detailed user guides
demonstrating the use of Bonsai.ML to solve state-of-the-art intelligent
experimental control problems.

% We will collaborate with experimental neuroscientists in using Bonsai.ML
% methods to address together some of their challenging problems in intelligent
% experimental control.
%
These collaborations will last between four and eight months, and are designed
to allow sufficient time to customise and optimise the Bonsai.ML methods for the
unique needs of the experimental collaborators, build new Bonsai.ML methods when
needed, and write high-quality use cases documenting outputs of the
collaborations.

The focus of these collaborations will be on the \emph{application} of
Bonsai.ML methods, and not on scientific research.
%
We have selected collaborations where Bonsai.ML methods can have a direct impact,
without requiring much scientific research, and where the impact can be clearly
demonstrated.

Outputs of these collaborations will be described in detailed uses cases in
Bonsai.ML's documentation.
%
In addition, when applications of Bonsai.ML's methods lead to new scientific
discoveries, details of these applications will be presented in research
publications.
%
However, even if a collaboration does not yield a new finding, the Bonsai.ML
use case will still be created, as it will describe in detail the application
of a Bonsai.ML method to a state-of-the-art neuroscience problem, and should
motivate Bonsai users to apply the method to related scientific problems.

Below we summarise the collaboration we will develop, which are detailed in the
Section~\ref{sec:approach} below.

\subsubsubsection*{Collaboration~1: real-time neural latents visualisation for
high-channel-count electrophysiology recordings with
linear dynamical systems}

\begin{description}

    \item[Collaborator:] Dr.~Josh Siegle, senior scientist, Allen Institute for
Neural Dynamics, US.

    \item[Summary:]

        Visualisation is central to scientific inquiry, especially in
        neuroscience. Yet, modern high-density recordings from hundreds or
        thousands of electrodes pose major challenges for clear, real-time
        visualisation. At the Gatsby Unit, we pioneered single-trial
        dimensionality reduction with latent-variable models~\citep{yuEtAl09},
        and extended these approaches in later work
        \citep[e.g.,][]{dunckerAndSahani18}, supported by open-source software
        (\href{https://github.com/joacorapela/svGPFA}{svGPFA}).  However, all
        current latent-variable methods operate offline, after data collection.
        In this project we will integrate state-space models into Bonsai.ML, to
        estimate and visualise latent variables online, during
        high-channel-count recordings.

    \item[Methods:] Gaussian linear dynamical system (already integrated),
Poisson linear dynamical system (to be integrated).

\end{description}

\subsubsubsection*{Collaboration~2: real-time forecasting animal behavior for
zero-lag stimulus presentation in augmented-reality small-animal experiments.}

\begin{itemize}

    \item\textbf{Collaborator:} Prof.~Aman Saleem,
        \href{https://www.saleemlab.com/}{Saleem Lab}, University College London,
        UK.

    \item\textbf{Summary:} In augmented reality experiments, particularly in
        neuroscience, precise stimulus timing is critical, yet unavoidable
        system delays mean that visual stimuli often appear slightly later than
        intended. This latency is especially problematic when linking neural
        activity to sensory input or behavior. To address this, we will use
        algorithms that forecast animal position and head orientation, allowing
        stimuli to be pre-rendered and displayed in synchrony with the
        subject’s actual position and head orientation.

    \item\textbf{Methods:} Gaussian linear dynamical system (already integrated),
deep learning forecasters (to be integrated).

\end{itemize}

\subsubsubsection*{Collaboration~3: real-time decoding of position from neural
spikes.}

\begin{itemize}

    \item\textbf{Collaborator:} Prof.~John O'Keefe and Prof.~Marcus Stephenson
        Jones, Sainsbury Wellcome Center, University College London, UK.

    \item\textbf{Summary:} We have already integrated into Bonsai.ML a fast
        implementation of a clusterless point-process decoder~\citep[i.e., a
        decoder that does not require spike sorting and is suitable for
        real-time usage,][]{denovellisEtAl21}. However, the current
        implementation of this decoder cannot decode unsorted spikes from
        Neuropixels probes.
        %
        This is a sever limitation, since Neuropixels are becoming the standard
        for high-channel-count electrophysiological recordings.
        %
        We will extend our implementation of the decoder to operate on
        Neuropixels recordings, and demonstrate this extension to decode rat
        positions from hippocampal spikes in a honeycomb
        maze~\citep{woodEtAl18}, and mice positions from striatal spikes in a
        sequence learning task~\citep{thompsonEtAl24}.

    \item\textbf{Methods:} Clusterless point-process decoder of animal position
        from hippocampal activity recorded with tetrodes recordings (already
        integrated), real-time clusterless point-process decoder of animal
        position from hippocampal recordings with Neuropixels probes
        (extension to Neuropixels to be developed; real-time constrains to be
        tested).

\end{itemize}

\subsubsection{New Bonsai.ML methods}

As part of collaboration~1, we will integrate into Bonsai.ML the \textbf{Poisson
linear dynamical systems} model, which infers latents assuming that spike
counts follow a Poisson distribution. This distribution is better for spike
count than the Gaussian distribution used in the Gaussian linear dynamical
system model already integrated into Bonsai.ML

For collaboration~2, we will evaluate, and possibly integrate into Bonsai.ML,
\textbf{deep learning forecasters}, as these forecasters are able to generate
superior long-horizon predictions than forecasters based on linear dynamical
models.

As part of collaboration 3, we will develop a \textbf{clusterless version of
the point-process decoder able to process recordings from Neuropixels probes}.

% \putbib
% \end{bibunit}

\pagebreak
\subsection{Approach}

\begin{instruction}

Word limit: 4,400

How are you going to deliver your proposed work?

What the assessors are looking for in your response

Explain how you have designed your approach so that it:

is effective and appropriate to achieve your objectives

\begin{itemize}

	\item is feasible, and comprehensively identifies any risks to delivery and
	how they will be managed

	\item uses a clearly written and transparent methodology (if applicable)

	\item summarises the previous work and describes how this will be built
	upon and progressed (if applicable)

	\item will maximise translation of outputs into outcomes and impacts

	\item describes how your, and if applicable your team’s, research
	environment (in terms of the place and relevance to the project) will
	contribute to the success of the work

\end{itemize}

Include the following when describing your approach:

\begin{itemize}

	\item measurable targets against which the outcome of the work will be
	assessed

	\item significant technical details for the development, maintenance or
	enhancement of the resource, indicating how this is of internationally
	exceptional quality

	\item data management approach, including plans for data collection,
	storage, curation, access, sharing, and long-term sustainability. Clearly
	outline how the resource will comply with relevant data standards, ensure
	interoperability, and support reuse by the wider research community.
	Information already included in the ‘Data management and sharing’ section
	does not need to be duplicated

	\item any proposed research efforts and how they directly facilitate
	development of the resource (if applicable)

	\item if the focus is on maintaining an existing resource instead of
	suggesting further development, provide evidence of why significant
	upgrades are not required at this time and detail why the resource needs
	continued support to maintain world-leading functionality (if applicable)

\end{itemize}

Describe the specific contribution of each applicant to the proposed resource:

\begin{itemize}

	\item their scientific contributions, for example, research field and
	specialist knowledge, experience, resource management expertise, technical
	and data analysis expertise

	\item their role and responsibilities, for example, managerial, leadership,
	mentoring

	\item references to specific work packages are recommended

	\item highlight where applicants will work collaboratively to deliver
	specific project requirements

	\item include clear time commitments for each applicant

\end{itemize}

There is no need to duplicate information included in the ‘Applicant and team
capability to deliver’ section.

References may be included within this section.

You may demonstrate elements of your responses in visual form if relevant. A
project Gantt chart identifying appropriate deliverables, responsibilities, and
time points for each objective, is recommended. Further details on uploading
images are provided in the Funding Service.

\end{instruction}

% \begin{bibunit}[unsrtveryabbrv]
% \subsection{Collaboration 3: Real-time decoding of replay from striatal spikes}

In the sequence learning task illustrated in
Fig.~\ref{fig:sequenceLearningTasks}, Dr.~Emmeett Thompson and Prof.~Marcus
Stephenson Jones and collaborators discovered replay in the dorsal striatum of
neural firing patterns that occurred during non-declarative, procedural,
experience~\citep{thomsonEtAl24}.
%
They demonstrated that striatal replay is needed for procedural memory
consolidation.
%
However, they could not establish if striatal replay is driving this
consolidation.
%
To address this question they need a fast method that can detect the initiation
of replay and disrupt it soon after it begins.

The spike sequence detection method~\citep{williamsEtAl20} used in
\citet{thompsonEtAl24} is an offline method. Thus, Dr.~Thompson and
Prof.~Stepenson Jones asked us if our
\href{https://bonsai-rx.org/machinelearning/examples/examples/PointProcessDecoder/DecodePositionFromHippocampusClusterless/README.html}{clusterless
hippocampal decoder} could be used to detect replay in the striatum with very
short latency.



be addressed in \citet{thompsonEtAl24} is
whether striatal replay contributes

% \pagebreak
% \putbib
% \end{bibunit}

\pagebreak
\subsection{Community demand: letters (or emails) of support}

\begin{instruction}
Letters (or emails) of support demonstrating community demand are mandatory
for BBR.

Upload a single PDF of maximum 8MB containing a maximum of 10 letters or
emails of support. These should be uploaded in English or Welsh only. Enter
the words ‘attachment supplied’ in the text box.

What the assessors are looking for in your response

The letters should give an indication of community demand for the resource in
question, demonstrating the breadth of research and the high-quality science
relevant to BBSRC remit that the resource would underpin.

Add the following details for each letter:

\begin{itemize}

	\item the organisation name (searchable via a drop-down list or enter the
organisation’s details manually, as applicable)

	\item contact name of the signatory

\end{itemize}

Letters of support aimed at demonstrating community demand should:

\begin{itemize}

	\item outline the uniqueness and expected added value of the proposed
resource to the UK bioscience research community and infrastructure landscape

	\item clearly explain the impact and benefit of the proposed resource on
the writer’s research and the associated community

	\item if possible, explain where this supported research has already
demonstrated or could have potential for particular scientific, economic or
societal impact

	\item help to demonstrate the breadth of the relevant user community

\end{itemize}

Letters of support that fail to do so, in particular template letters
indicating generic support without identifying a particular usage, are of
negligible value for the assessment and should not be submitted. Carefully
chosen letters containing relevant evidence of the requirement or benefit to be
gained, are of greater value than large numbers of letters.

The Funding Service will provide document upload details when you apply.

\end{instruction}

\pagebreak
\subsection{Applicant and team capability to deliver}

\begin{instruction}
Word limit: 1,650

Why are you the right individual or team to successfully deliver the proposed
work?

What the assessors are looking for in your response

Please ensure the current job titles of the core team members are included
here to ensure eligibility can be established for the core team roles assigned.
Find out more about
\href{https://www.ukri.org/publications/roles-in-funding-applications/roles-in-funding-applications-eligibility-responsibilities-and-costings-guidance/}{UKRI’s
core team roles in funding applications} and our
\href{https://www.ukri.org/councils/bbsrc/guidance-for-applicants/check-if-youre-eligible-for-funding/applicants-and-co-applicants/}{eligibility
guidance}.

Evidence of how you, and if relevant your team, have:

\begin{itemize}

    \item the relevant experience (appropriate to career stage) to deliver the proposed
work

    \item the right balance of skills and expertise to cover the proposed work

    \item the appropriate leadership and management skills to deliver the work and
your approach to develop others

    \item contributed to developing a positive research environment and wider
community

\end{itemize}

You may demonstrate elements of your responses in visual form if relevant.

Further details are provided in the Funding Service.

The word limit for this section is 1,650 words: 1,150 words to be used for R4RI
modules (including references) and, if necessary, a further 500 words for
Additions.

Use the Résumé for Research and Innovation (R4RI) format to showcase the range
of relevant skills you and, if relevant, your team (project and project
co-leads, researchers, technicians, specialists, partners and so on) have and
how this will help deliver the proposed work. You can include individuals’
specific achievements but only choose past contributions that best evidence
their ability to deliver this work.

Complete this section using the R4RI module headings listed. Use each heading
once and include a response for the whole team, see the
\href{https://www.ukri.org/apply-for-funding/before-you-apply/resume-for-research-and-innovation-r4ri-guidance/}{UKRI
guidance on R4RI}. You should consider how to balance your answer, and
emphasise where appropriate the key skills each team member brings:

\begin{itemize}

	\item contributions to the generation of new ideas, tools, methodologies, or
knowledge

	\item the development of others and maintenance of effective working relationships

	\item contributions to the wider research and innovation community

	\item contributions to broader research or innovation users and audiences and
towards wider societal benefit

\end{itemize}

Additions

Provide any further details relevant to your application. This section is
optional and can be up to 500 words. You should not use it to describe
additional skills, experiences, or outputs, but you can use it to describe any
factors that provide context for the rest of your R4RI (for example, details of
career breaks if you wish to disclose them).

Complete this as a narrative. Do not format it like a CV.

There is no need to duplicate information included in the ‘Approach’ section.

References may be included within this section.

The roles in funding applications policy has descriptions of the different
project roles.

\end{instruction}

% \input{team}

\pagebreak
\subsection{Project partners}

\begin{instruction}

Add details about any project partners’ contributions. If there are no project
partners, you can indicate this on the Funding Service.

A project partner is a collaborating organisation who will have an integral
role in the proposed research. This may include direct (cash) or indirect
(in-kind) contributions such as expertise, staff time or use of facilities.
Project partners may be in industry, academia, third sector or government
organisations in the UK or overseas, including partners based in the EU.

If you are applying via the IPA or LINK scheme, please include details of
industry partners here.

If applying under the BBSRC-NSF lead agency scheme, please include details
of your US partner here.

Add the following project partner details:

\begin{itemize}

    \item the organisation name and address (searchable via a drop-down list or enter
the organisation’s details manually, as applicable)

    \item the project partner contact name and email address

    \item the type of contribution (direct or in-direct) and its monetary value

\end{itemize}

If a detail is entered incorrectly and you have saved the entry, remove the
specific project partner record and re-add it with the correct information.

For audit purposes, UKRI requires formal collaboration agreements to be put in
place if an award is made.

\end{instruction}

% \input{projectPartners}

\pagebreak
\subsection{Project partners: statement of support}

\begin{instruction}

Word limit: 3,000

Upload a single PDF containing the letters or emails of support from each
partner you named in the ‘Project partners’ section. These should be uploaded
in English or Welsh only.

What the assessors are looking for in your response

Enter the words ‘attachment supplied’ in the text box, or if you do not have any
project partners enter ‘N/A’. Each statement should:

\begin{itemize}

    \item confirm the partner’s commitment to the project

    \item clearly explain the value, relevance, and possible benefits of the
    work to them

    \item describe any additional value that they bring to the project

	\item have a page limit of two sides A4 per partner

\end{itemize}

The Funding Service will provide document upload details when you apply.

If you do not have any project partners, you will be able to indicate this in
the Funding Service.

Ensure you have prior agreement from project partners so that, if you are
offered funding, they will support your project as indicated in the ‘Project
partners’ section.

For audit purposes, UKRI requires formal collaboration agreements to be put in
place if an award is made.

\end{instruction}
% \input{projectPartnersStmtOfSupport}

\pagebreak
\subsection{Management strategy}

\begin{instruction}

Word limit: 500

How do you plan to manage the resource?

What the assessors are looking for in your response

Provide details on:

\begin{itemize}

	\item governance arrangements appropriate for the oversight and successful
delivery of the project’s complexity

	\item the project’s management and advisory structure

	\item the approach to project and risk management, and the monitoring
strategy for tracking progress of the proposed programme

	\item how demand and access requests will be managed, and what support will
be provided to the users of the resource

	\item the proposed membership of the advisory board and how it will be
used. An advisory board is required and is to be independent from both the
academic institutions and project partners involved in the proposal

	\item how the resource user perspective and their needs will be considered,
including how feedback will be sought and subsequently used to inform the
management of the resource

\end{itemize}

You may demonstrate elements of your responses in visual form if relevant.
Further details are provided in the Funding Service.

\end{instruction}

\pagebreak
\subsection{Data management and sharing}

\begin{instruction}

Word limit: 1,500

How will you manage and share data collected or acquired through the
proposed resource?

What the assessors are looking for in your response

Provide a data management plan using the BBR DMP template structure,
provided in the ‘Supporting documents’ section, that clearly details how you will
comply with UKRI’s published data sharing policy, which includes detailed
guidance notes.

\end{instruction}

% \input{dataManagement}

\pagebreak
\subsection{Resources and cost justification}

\begin{instruction}

Word limit: 1,000

What will you need to deliver your proposed work and how much will it cost?

What the assessors are looking for in your response

The FEC of your project can be up to a maximum of £2 million. We will fund
80\% of the FEC. For example, if the FEC cost of your project is equal £2
million, we will fund £1.6 million and your research organisation will be
expected to fund £400,000.

Please note, equipment over £10,000 is funded by BBSRC at 50\%. The
Funding Service does not currently have the ability to record this. For this round
we ask that you include equipment over £10,000 in ‘Exceptions’ at 100\% of
cost. We will cut this to 50\% at award. You must ensure you have prior
agreement from your research organisation to fund the remaining 50\%.

Justify the application’s more costly resources, in particular:

\begin{itemize}

	\item project staff significant

	\item travel for field work or collaboration (but not regular travel
between collaborating organisations or to conferences)

	\item any equipment that will cost more than £10,000

	\item any consumables beyond typical requirements, or that are required in
exceptional quantities

	\item all facilities and infrastructure costs

	\item all resources that have been costed as ‘Exceptions’

\end{itemize}

Assessors are not looking for detailed costs or a line-by-line breakdown of all
project resources. Overall, they want you to demonstrate how the resources
you anticipate needing for your proposed work:

\begin{itemize}

	\item are comprehensive, appropriate, and justified

	\item represent the optimal use of resources to achieve the intended
outcomes

	\item maximise potential outcomes and impacts

\end{itemize}

\end{instruction}

% \input{resources}

\pagebreak
\subsection{Trusted research and innovation (TR\&I)}

\begin{instruction}

Word limit: 500

Does your proposed work relate to UKRI’s Trusted Research and Innovation
principles?

What the assessors are looking for in your response

Demonstrate how your proposed work relates to UKRI’s Trusted Research and
Innovation principles including:

\begin{itemize}

	\item list any dual-use (both military and non-military) applications to your research

	\item if this project is relevant to one or more of the 17 areas of the UK National
Security and Investment (NSI) Act, please list the area(s)

	\item please read the academic export control guidance and confirm if an export
control licence is required for this project and the status of any application(s)

	\item if your project involves any items or substances on the UK strategic export
control list, please provide a list

\end{itemize}

In this context, you should be aware that the UK National Security \& Investment
Act, 2021 includes Data Infrastructure, and other biosciences-relevant areas,
within the 17 sensitive areas of the economy.

We may ask you to provide additional TR\&I information later, in line with UKRI
TR\&I principles and funding terms and conditions (RGC 2.6.2, 2.7.1 and 2.7.2).

\end{instruction}

% \input{trustedResearchAndInnovation}

\pagebreak
\subsection{Ethics and responsible research and innovation (RRI)}

\begin{instruction}

Word limit: 500

What are the ethical or RRI implications and issues relating to the proposed
work? If you do not think that the proposed work raises any ethical or RRI
issues, explain why.

What the assessors are looking for in your response

Demonstrate that you have identified and evaluated:

\begin{itemize}

	\item the relevant ethical or responsible research and innovation considerations

	\item how you will manage these considerations

\end{itemize}

If you are collecting or using data you should identify:

\begin{itemize}

	\item any legal and ethical considerations of collecting, releasing or
storing the data (including consent, confidentiality, anonymisation, security
and other ethical considerations and, in particular, strategies to not preclude
further reuse of data)

	\item formal information standards that your proposed work will comply with

\end{itemize}

Additional sub-questions (to be answered only if appropriate) relating to
research involving:

\begin{itemize}

	\item animals

	\item human participants

	\item genetically modified organisms

\end{itemize}

You may demonstrate elements of your responses in visual form if relevant.
Further details are provided in the Funding Service.

\end{instruction}

% \input{rri}

\pagebreak
\subsection{Genetic and biological risk}

\begin{instruction}

Word limit: 700

Does your proposed research involve any genetic or biological risk?

What the assessors are looking for in your response

In respect of animals, plants or microbes, are you proposing to:

\begin{itemize}

	\item use genetic modification as an experimental tool, like studying gene
function in a genetically modified organism

	\item release genetically modified organisms

	\item ultimately develop commercial and industrial genetically modified
outcomes

\end{itemize}

If yes, provide the name of any required approving body and state if approval
is already in place. If it is not, provide an indicative timeframe for
obtaining the required approval.

Identify the organism or organisms as a plant, animal or microbe and specify
the species and which of the three categories the research relates to.

Identify the genetic and biological risks resulting from the proposed research,
their implications, and any mitigation you plan on taking. Assessors will want
to know you have considered the risks and their implications to justify that
any identified risks do not outweigh any benefits of the proposed research.

If this does not apply to your proposed work, you will be able to indicate this
in the Funding Service.

\end{instruction}

\pagebreak
\subsection{Research involving the use of animals}

\begin{instruction}

Does your proposed research involve the use of vertebrate animals or other
organisms covered by the Animals Scientific Procedures Act?

What the assessors are looking for in your response

If you are proposing research that requires using animals, download and
complete the Animals Scientific Procedures Act template (DOCX, 74KB),
which contains all the questions relating to research using vertebrate animals or
other Animals (Scientific Procedures) Act 1986 regulated organisms.

Save it as a PDF, ensuring it is no larger than 8MB. The Funding Service will
provide document upload details when you apply.

If this does not apply to your proposed work, you will be able to indicate
this in the Funding Service.

\end{instruction}

\pagebreak
\subsection{Conducting research with animals overseas}

\begin{instruction}

Word limit: 700

Will any of the proposed animal research be conducted overseas?

What the assessors are looking for in your response

If you are proposing to conduct overseas research, it must be conducted in
accordance with welfare standards consistent with those in the UK, as in
Responsibility in the use of animals in bioscience research. Ensure all
named applicants in the UK and overseas are aware of this requirement.
Ensure all named applicants in the UK and overseas are aware of this
requirement.

If your application proposes animal research to be conducted overseas, you
must provide a statement in the text box. Depending on the species involved,
you may also need to upload a completed template for each species listed.

Statement

Provide a statement to confirm that:

\begin{itemize}

	\item all named applicants are aware of the requirements and have agreed to
abide by them

	\item this overseas research will be conducted in accordance with welfare
standards consistent with the principles of UK legislation

	\item the expectation set out in Responsibility in the use of animals in bioscience
research will be applied and maintained

	\item appropriate national and institutional approvals are in place

\end{itemize}

Templates

Overseas studies proposing to use non-human primates, cats, dogs, equines or
pigs will be assessed during NC3Rs review of research applications. Provide
the required information by completing the template from the question
‘Research involving the use of animals’.

For studies involving other species, such as:

\begin{itemize}

	\item rodents

	\item rabbits

	\item sheep

	\item goats

	\item pigs

	\item cattle

	\item xenopus laevis and xenopus tropicalis

	\item zebrafish

\end{itemize}

Select, download, and complete the relevant Word checklist or checklists by
exploring NC3Rs checklist for the use of animals overseas.

Save your completed template as a PDF and upload to the Funding Service. If
you use more than one checklist template, save it as a single PDF.
The Funding Service will provide document upload details when you apply.

If conducting research with animals overseas does not apply to your proposed
work, you will be able to indicate this in the Funding Service.

\end{instruction}

\pagebreak
\subsection{Research involving human participation}

\begin{instruction}

Word limit: 700

Will the project involve the use of human subjects or their personal information?
What the assessors are looking for in your response

If you are proposing research that requires the involvement of human subjects,
provide the name of any required approving body and whether approval is
already in place.Justify the number and the diversity of the participants involved, as well as any
procedures.

Provide details of any areas of substantial or moderate severity of impact.

If this does not apply to your proposed work, you will be able to indicate this in
the Funding Service.

\end{instruction}

\pagebreak
\subsection{Research involving human tissues or biological samples}

\begin{instruction}

Word limit: 700

Does your proposed research involve the use of human tissues, or biological
samples?

What the assessors are looking for in your response

If you are proposing work that involves human tissues or biological samples,
provide the name of any required approving body and whether approval is
already in place.

Justify the use of human tissue or biological samples specifying the nature and
quantity of the material to be used and its source.

If this does not apply to your proposed work, you will be able to indicate this in
the Funding Service.

\end{instruction}

\end{document}
