% Bonsai

\href{https://bonsai-rx.org/}{Bonsai} is a free and open-source visual reactive
programming language widely used for experimental control in neuroscience.
Designed for performance, flexibility, and ease of use, Bonsai enables
scientists with little or no programming background to build high-performance
data acquisition and control systems. With more than 7,000 downloads annually,
nearly 100 citations per year of the core paper, and over 1,000 new users in
2024 alone, Bonsai has become the most widely adopted software platform in
systems neuroscience. Its success demonstrates the transformative role that
sustainable, community-driven research software can play in accelerating
discovery.

% Bonsai.ML
% BBSRC grant

A central priority of UKRI is the application of
\href{https://www.ukri.org/what-we-do/browse-our-areas-of-investment-and-support/artificial-intelligence-in-bioscience/}{artificial
intelligence in bioscience}. In 2022 we recognised that integrating machine
learning (ML) into Bonsai could be transformative for experimental
neuroscience. With BBSRC support
(\href{https://gow.bbsrc.ukri.org/grants/AwardDetails.aspx?FundingReference=BB\%2FW019132\%2F1}{BB/W019132/1}),
we developed Bonsai.ML, which extends Bonsai with state-of-the-art ML methods.
These include
\href{https://bonsai-rx.org/machinelearning/examples/examples/LinearDynamicalSystems/README.html}{Linear
Dynamical Systems},
\href{https://bonsai-rx.org/machinelearning/examples/examples/HiddenMarkovModels/README.html}{Hidden
Markov Models},
\href{https://bonsai-rx.org/machinelearning/examples/examples/Torch/NeuralNetsTrainedOnline/README.html}{Deep
Neural Networks}, and a
\href{https://bonsai-rx.org/machinelearning/examples/examples/PointProcessDecoder/DecodePositionFromHippocampusSortedUnits/README.html}{Point-Process
Decoder}. Embedding these models directly in Bonsai's reactive programming
environment enables adaptive, data-driven experimental designs that were
previously out of reach for many laboratories.

% Challenges

However, adding even the best ML methods to Bonsai.ML is not enough to guarantee
their uptake, since most Bonsai users have little or no training in ML, and
cannot immediately appreciate how these methods can aid their experimentation.
%
% This challenge is not unique to Bonsai.ML, but is common to all software
% seeking to provide ML functionality to non-ML-specialists.
%
% If successful, our proposed solution will serve as a role model to all these
% software.

% Solutions

We will accelerate the uptake and impact of Bonsai.ML through two complementary
strategies.
%
First, we will substantially revamp Bonsai.ML's documentation and strengthen
the current training and dissemination activities. This will empower
experimental neuroscientists to explore and identify ``killer applications''
of Bonsai.ML to neuroscience independently.
%
Second, we will work hand-in-hand with experimental neuroscientists to address
challenging neuroscience problems with Bonsai.ML, and together demonstrate
transformative applications of Bonsai.ML to neuroscience experimental control.
%
% By reading or watching use cases of applications of Bonsai.ML methods in
% intelligent experimental control, experimental neuroscientists could realise the
% utility of these methods for their own experiments and adopt Bonsai.ML.

\subsubsection{Collaborations}

Collaborations with experimental groups are essential for the creation of
machine learning methods targeted to biological applications. These
collaborations benefit both the experimental and the methods development
partners.
%
The former partner benefits with the machine learning expertise of the methods
partner to interrogate its experimental data.
%
The latter partner benefits by having access to state-of-the-art experimental
data, and with expertise on the significance and potential of applications of
the machine learning method in biological problems.
%
These collaborations are essential to maximise the power of
\href{https://www.ukri.org/what-we-do/browse-our-areas-of-investment-and-support/artificial-intelligence-in-bioscience/}{artificial
intelligence in bioscience}.

During the creation of Bonsai.ML, we attempted a few collaborations with
experimental neuroscience groups as, from our experience disseminating Bonsai,
we know that these collaborations are essential to maximize uptake of software
for neuroscientists.
%
In some cases we obtained promising initial results, but achieving the full
goals of the collaboration required further developments, that we did not have
time to pursue.
%
In other collaborations, initial results were not satisfactory, but we did not
dedicated further time to understand the problems that yielded poor results,
since we were focused on integrating machine learning functionality into
Bonsai.ML.

Now that Bonsai.ML has been created, and that it contains core real-time
machine learning methods, we aim at developing collaborations with experimental
neuroscience groups, use methods already integrated into Bonsai.ML to jointly
tackle their intelligent experimental control problems, and use the results of
this collaboration to create a detailed user guides demonstrating the use of
Bonsai.ML to solve state-of-the-art intelligent experimental control problems.

% We will collaborate with experimental neuroscientists in using Bonsai.ML
% methods to address together some of their challenging problems in intelligent
% experimental control.
%
These collaborations will last between four and eight months, and are designed
to allow sufficient time to customise and optimise the Bonsai.ML methods for the
unique needs of the experimental collaborators, build new Bonsai.ML methods when
needed, and write high-quality use cases documenting outputs of the
collaborations.

The focus of these collaborations will be on the \emph{application} of
Bonsai.ML methods, and not on scientific research.
%
We have selected collaborations where Bonsai.ML methods can have a direct impact,
without requiring much scientific research, and where the impact can be clearly
demonstrated.

Outputs of these collaborations will be described in detailed uses cases in
Bonsai.ML's documentation.
%
In addition, when applications of Bonsai.ML's methods lead to new scientific
discoveries, details of these applications will be presented in research
publications.
%
However, even if a collaboration does not yield a new finding, the Bonsai.ML
use case will still be created, as it will describe in detail the application
of a Bonsai.ML method to a state-of-the-art neuroscience problem, and should
motivate Bonsai users to use the method in other scientific problems.

Below we summarise the collaboration we will develop, which are detailed in the
Section~\ref{sec:approach} section.

\subsubsubsection*{Collaboration~1: real-time neural latents visualisation}

\paragraph{Collaborator:} Josh Siegle, senior scientist, Allen Institute for
Neural Dynamics, US.

\paragraph{Methods:} Gaussian linear dynamical system (already integrated),
Poisson linear dynamical system (to be integrated).

\paragraph{Duration:} 5 months.
