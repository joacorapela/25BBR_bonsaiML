\subsection{Collaboration 3: Real-time decoding of replay from striatal spikes}

In the sequence learning task illustrated in
Fig.~\ref{fig:sequenceLearningTasks}, Dr.~Emmeett Thompson and Prof.~Marcus
Stephenson Jones and collaborators discovered replay in the dorsal striatum of
neural firing patterns that occurred during non-declarative, procedural,
experience~\citep{thomsonEtAl24}.
%
They demonstrated that striatal replay is needed for procedural memory
consolidation.
%
However, they could not establish if striatal replay is driving this
consolidation.
%
To address this question they need a fast method that can detect the initiation
of replay and disrupt it soon after it begins.

The spike sequence detection method~\citep{williamsEtAl20} used in
\citet{thompsonEtAl24} is an offline method. Thus, Dr.~Thompson and
Prof.~Stepenson Jones asked us if our
\href{https://bonsai-rx.org/machinelearning/examples/examples/PointProcessDecoder/DecodePositionFromHippocampusClusterless/README.html}{clusterless
hippocampal decoder} could be used to detect replay in the striatum with very
short latency.



be addressed in \citet{thompsonEtAl24} is
whether striatal replay contributes
